\section{Лекция 3 10.03.25}

\subsection*{Задача интерполяции}
$[a, b] \subset \mathbb{R}$; $x_i \in [a, b]$ (различные точки) --- узлы интерполяции \\
Множество $\{x_1, x_2, \dots, x_n\}$ --- сетка \\
Даны $y_i$, $i = 0, 1, \dots, n$ \\
Найти $g(x) \in C[a, b]$, т. ч. $g(x_i) = y_i$ --- интерполяционная функция или интерполянт \\
Часто $y_i = f(x_i)$ --- тогда интерполяция функции $f$ на сетке $x$

\subsection*{Опр. (Алг. интерполяция)}
Интерполяция с помощью алгебраических полиномов --- алгебраическая интерполяция 
\[ 
P_m(x) = a_0 + a_1 x + \dots + a_m x^m;
\]
алгебраический интерполяционный полином --- алгебраический интерполянт, решающий задачу.

\subsubsection*{Как найти коэффициенты $a_0, a_1, \dots, a_m$?}
Условия интерполяции:
\[ P_m(x_i) = y_i \]
\[
\begin{cases} 
a_0 + a_1 x_0 + a_2 x_0^2 + \dots + a_m x_0^m = y_0 \\ 
a_0 + a_1 x_1 + a_2 x_1^2 + \dots + a_m x_1^m = y_1 \\ 
\vdots \\
a_0 + a_1 x_n + a_2 x_n^2 + \dots + a_m x_n^m = y_n 
\end{cases}
\] \\
При различных $m$, $n$ может не быть решения или быть много решений.

\subsection*{Теорема}
При $m = n$ решение задачи алгебраической интерполяции существует и единственно.

\begin{proof}
Матрица системы имеет вид:
\[ V(x_0, x_1, \dots, x_n) = \begin{pmatrix} 
1 & x_0 & x_0^2 & \dots & x_0^n \\ 
1 & x_1 & x_1^2 & \dots & x_1^n \\ 
\vdots & \vdots & \vdots & \ddots & \vdots \\ 
1 & x_n & x_n^2 & \dots & x_n^n 
\end{pmatrix} \text{--- матрица Вандермонда} \] \\
Определитель матрицы Вандермонда:
\[ \det V = \prod_{0 \leq i < j \leq n} (x_i - x_j) \neq 0, ~ \text{т.к. узлы различны} \] 
\[ \Rightarrow \text{СЛАУ однозначно разрешима} \]
\end{proof}

\subsection*{Проблемы:}
\begin{enumerate}
    \item большая вычислительная сложность решения СЛАУ;
    \item матрица Вандермонда чувствительна к возмущениям ("плохо обусловлена").
\end{enumerate}

\subsubsection*{Но:}
\begin{itemize}
    \item нам редко нужна каноническая форма;
    \item нам важнее просто уметь вычислять значения в любой точке.
\end{itemize}

Пусть $P(x)$ решает задачу интерполяции по $y = (y_0, y_1, \ldots, y_n)$, ~ $Q(x)$ по $z = (z_0, z_1, \ldots, z_n)$. \\
Тогда $\forall \alpha, \beta \in \mathbb{R} \quad \alpha P(x) + \beta Q(x)$ решает задачу интерполяции по $\alpha y + \beta z$ на тех же узлах, т. е. "оператор интерполяции линеен".

Соберём вместе многочлены, удовлетворяющие отдельным интерполяционным условиям:
\[
\sqsupset P_n(x) = \sum_{i=0}^n y_i \varphi_i(x), \quad
\text{где} ~ \varphi_i(x): \deg \varphi_i(x) = n, ~ \varphi_i(x_j) =
\begin{cases}
0, & i \neq j \\
1, & i = j
\end{cases}
\]

$y_i \varphi_i(x)$ --- решение задачи интерполяции для $(0, \ldots, 0, y_i, 0, \ldots, 0)$ на узлах $x_0, x_1, \ldots, x_n$, \\
$\deg (y_i \varphi_i(x)) = n$; \\
$\Rightarrow \sum_{i=0}^n y_i \varphi_i(x)$ решает задачу интерполяции для $(y_0, y_1, \ldots, y_n)$.

\subsection*{Как найти $\varphi_i(x)$?}

Ноль в точках $x_0, \ldots, x_{i-1}, x_{i+1}, \ldots, x_n$
\[
\varphi_i(x) = K_i (x - x_0) \ldots (x - x_{i-1})(x - x_{i+1}) \ldots (x - x_n), \quad K_i \in \mathbb{R}
\]
т. к. $\deg \varphi_i(x) = n$. Т. к. $\varphi(x_i) = 1$
\[
K_i =  \frac{1}{(x_i - x_0)(x_i - x_1) \ldots (x_i - x_{i-1})(x_i - x_{i+1}) \ldots (x_i - x_n)}
\]

\[
\varphi_i(x) = \frac{(x - x_0)(x - x_1) \ldots (x - x_{i-1})(x - x_{i+1}) \ldots (x - x_n)}{(x_i - x_0)(x_i - x_1) \ldots (x_i - x_{i-1})(x_i - x_{i+1}) \ldots (x_i - x_n)}
\]

\[
P_n(x) = \sum_{i=0}^n y_i \frac{\prod_{j \neq i} (x - x_j)}{\prod_{j \neq i} (x_i - x_j)} \text{ --- интерполяционный полином Лагранжа}
\]

\[
\sqsupset w_n(x) = (x - x_0)(x - x_1) \ldots (x - x_{i-1})(x - x_i)(x - x_{i+1}) \ldots (x - x_n), \quad (\deg w_n(x) = n + 1)
\]

Тогда
\[
\varphi_i(x) = \frac{ \frac{w_n(x)}{(x - x_i)}}{w_n'(x_i)} = \frac{w_n(x)}{(x - x_i)w_n'(x_i)}
\]

\[
P_n(x) = \sum_{i=0}^n y_i \frac{w_n(x)}{(x - x_i)w_n'(x_i)}
\]

\subsubsection*{Трудности:} Базисные полиномы зависят от всех узлов интерполяции сразу $\Rightarrow$ при смене одного узла нужно всё пересчитывать заново.

\subsection*{Разделённые разности}

\[ x_0 \to f(x_0) \]
\[ x_1 \to f(x_1) \]
\[ \ldots \]
\[ x_n \to f(x_n) \]
$x_0, \ldots, x_n$ --- различные точки.

\begin{itemize}
    \item разделённая разность первого порядка:
\[
f^{<}(x_i; x_{i+1}) := \frac{f(x_{i+1}) - f(x_i)}{x_{i+1} - x_i}
\]
    \item разделённая разность второго порядка:
\[
f^{<}(x_i; x_{i+1}; x_{i+2}) := \frac{f^{<}(x_{i+1}; x_{i+2}) - f^{<}(x_i; x_{i+1})}{x_{i+2} - x_i}
\]
    \item разделённая разность $k$-го порядка:
\[
f^{<}(x_i; x_{i+1}; \ldots; x_{i+k}) = \frac{f^{<}(x_{i+1}; \ldots; x_{i+k}) - f^{<}(x_i; \ldots; x_{i+k-1})}{x_{i+k} - x_i}
\]
\end{itemize}

Можно считать, что $f^{<}(x_i) = f(x_i)$ --- разделённая разность 0-го порядка совпадает со значением функции. 

\subsubsection*{Геометрический смысл:}
\begin{itemize}
    \item первый порядок --- тангенс наклона касательной;
    \item далее --- средняя скорость изменения. \quad // производная --- мгновенная скорость
\end{itemize}

\subsection*{Утв.}
$f^{<}(x_i; x_{i+1}; \ldots; x_{i+k}) = \sum_{j=i}^{i+k} \frac{f(x_j)}{\prod_{l = i, l \neq j}^{i+k} (x_j - x_l)}$

\begin{proof}
Индукция по $k$, много страшных вкладок, не будем.
\end{proof}

\subsection*{Сл-е:}
Разделённая разность --- симметричная функция своих аргументов, т. е. она не изменяется при любой их перестановке.

\subsection*{Интерполяционный полином Ньютона}

$P_k(x)$ --- интерполяционный полином по $x_0, x_1, \ldots, x_k$ \\
$P_0(x) = y_0$, тогда

\[
P_n(x) = P_0(x) + \sum_{k=1}^n ( P_k(x) - P_{k-1}(x))
\]
$\Rightarrow$ При добавлении/удалении узлов меняются только последние слагаемые.

Необходимо найти удобные представления разностей $P_k(x) - P_{k-1}(x)$: \\
$\deg \left( P_k(x) - P_{k-1}(x) \right) = k$ \\
$(P_k(x) - P_{k-1}(x))$ обращаются в ноль в $x_0, x_1, \ldots, x_{k-1} \Rightarrow$
\[
P_k(x) - P_{k-1}(x) = A_k (x - x_0)(x - x_1) \ldots (x - x_{k-1}), \quad A_k \in \mathbb{R}
\]
$P_k(x_k) = y_k$

\[
A_k = \frac{y_k - P_{k-1}(x_k)}{(x_k - x_0)(x_k - x_1) \ldots (x_k - x_{k-1})} = \frac{y_k - P_{k-1}(x_k)}{\prod_{l=0}^{k-1} (x_k - x_l)}
\]

Подставим вместо $P_{k-1}(x_k)$ выражения для инт. пол. в ф. Лагранжа, проделав страшные выкладки получаем
\[
A_k = \sum_{j=0}^k \frac{y_j}{\prod_{l=0, l \neq j}^k (x_j - x_l)} = (y_0,y_1, \ldots, y_k)^{<}
\]

\[
P_n(x) = y_0 + (y_0, y_1)^{<}(x-x_1) + (y_0,y_1,y_2)^{<}(x - x_0)(x - x_1) + \ldots
\]
\[
\ldots + (y_0, y_1, \ldots, y_n)^{<}(x - x_0)(x - x_1) \ldots (x - x_{n-1})
\]

Если $y_i = f(x_i)$:
\[
P_n(x) = f(x_0) + f^{<}(x_0; x_1)(x - x_0) + \ldots + f^{<}(x_0, x_1, \ldots, x_n)(x - x_0)(x - x_1) \ldots (x - x_{n-1})
\]
\[
 \text{ --- интерполяционный полином в ф. Ньютона}
\]

\subsection*{Частный случай}

Равномерная сетка: $h_i = x_{i+1} - x_i = \text{const} = h$ --- шаг сетки
\[
\Delta f(x) = f(x + h) - f(x)
\]
\[
\Delta^2 f(x) = \Delta(\Delta f(x))
\]

Можно показать, что $f^{<}(x_0, x_1, \ldots, x_k) = \frac{\Delta^k f(x_0)}{k! h^k} \quad k=1,2,\ldots$ \\
Тогда интерполяционный полином Ньютона имеет вид
\[
P_n(x) = f(x_0) + \frac{\Delta f(x_0)}{h}(x - x_0) + \frac{1}{2!}\frac{\Delta^2 f(x_0)}{h^2}(x - x_0)(x - x_1) + \ldots
\]
\[
\ldots + \frac{1}{n!}\frac{\Delta^n f(x_0)}{h^n}(x - x_0)(x - x_1) \cdots (x - x_{n-1})
\]

\subsection*{Погрешность алгебраической интерполяции}

$f$ --- функция, $g(x)$ --- её интерполянт; $R(f, x) = f(x) - g(x)$ --- остаток

\subsubsection*{Утв. 1}
Если $z$ не является точкой сетки, то \textit{(именно наличие формы Ньютона помогает нам это доказать)}

\[
R_n(f; z) = f^{<}(x_0, x_1, \ldots, x_n, z) w_n(z), \quad \text{где} ~ w_n(x) = (x - x_0)(x - x_1) \ldots (x - x_n)
\]

\begin{proof}
Выпишем для $f$ инт. полином Ньютона по $x_0, x_1, \ldots, x_n, z$
\[
P_{n+1}(x) = P_n(x) + f^{<}(x_0, x_1, \ldots, x_n, z)(x - x_0)(x - x_1) \cdots (x - x_n)
\]
$\sqsupset x = z$:

\[
P_{n+1}(z) = P_n(z) + f^{<}(x_0, x_1, \ldots, x_n; z)(z - x_0)(z - x_1) \cdots (z - x_n)
\]
Но $P_{n+1}(z) = f(z)$ по построению $P_{n+1}$
\[
\Rightarrow R_n(f, z) = f(z) - P_n(z) = f^{<}(x_0, x_1, \ldots, x_n, z)(z - x_0)(z - x_1) \cdots (z - x_n)
\]
\end{proof}

\begin{itemize}
    \item умеем находить погрешность в конкретной точке
    \item а что делать на всём интервале?
\end{itemize}

\subsubsection*{Утв 2.}
$f \in C^{n+1}[a, b]$; $\underline{x} = \min\{x_0, x_1, \ldots, x_n\}$, $\overline{x} = \max\{x_0, x_1, \ldots, x_n\}$. Тогда

\[
f^{<}(x_0, x_1, \ldots, x_n) = \frac{1}{n!} f^{(n)}(\xi), \quad \text{где} ~ \xi \in [\underline{x}, \overline{x}]
\]

\begin{proof}
\begin{enumerate}
\item Разделённая разность первого порядка:
\[
f^{<}(x_0, x_1) = \frac{f(x_1) - f(x_0)}{x_1 - x_0}
\]
По т. Лагранжа\footnote{\url{https://ru.wikipedia.org/wiki/Формула_конечных_приращений}} $\exists ~ \xi: f^{<}(x_0, x_1) = f'(\xi)$.

\item $\sqsupset \Theta(x) = f^{(n)}(x) - n! f^{<}(x_0, x_1, \ldots, x_n)$. Нужно доказать, что $\exists \xi : \Theta(\xi) = 0$:

$\sqsupset P_n(x)$ интерполяционный полином по $x_0, x_1, \ldots, x_n \Rightarrow \Theta(x) = f^{(n)}(x) - P^{(n)}_n(x) = R^{(n)}_n(f,x)$. \\
В узлах интерполяции $x_0, x_1, \ldots, x_n ~ R_n(f; x_i) = 0 \Rightarrow$ (по теореме Ролля\footnote{\url{https://ru.wikipedia.org/wiki/Теорема_Ролля}}) на каждом интервале $\exists p_i : R'_n(f, p_i) = 0$ \\
$\Rightarrow R'_n$ имеет n нулей $\Rightarrow R''_n$ имеет (n-1) нулей $\Rightarrow \ldots ~ R^{(n)}_n$ имеет ноль.
\end{enumerate}
\end{proof}

\subsection*{Теорема (о погрешности алгебраической интерполяции)}

$f \in C^{n+1}[a, b]$; $x_0, x_1, \ldots, x_n \in [a, b]$ — различные точки. Тогда справедливо представление:
\[
R_n(f; x) = \frac{f^{(n+1)}(\xi(x))}{(n+1)!} w_n(x), \quad \xi(x) \in [a, b] ~ \text{и зависит от} ~ x
\]

\begin{proof}
\[
\sqsupset x = x_i \Rightarrow w_n(x) = 0 \Rightarrow R_n(f; x_i) = 0 \Rightarrow \xi(x) \text{--- любая}
\]

\[
\sqsupset x \neq x_i \Rightarrow \text{(утв. 1)} ~ R_n(f; x) = f^{<}(x_0, x_1, \ldots, x_n, x) \cdot w_n(x)
\]

\[
\Rightarrow \text{(утв. 2)} ~ R_n(f,x) = \frac{f^{(n+1)}(\xi(x))}{(n+1)!} w_n(x)
\]
\end{proof}

$\sqsupset M_n = \max_{\xi \in [a, b]} |f^{(n)}(\xi)| \Rightarrow$
\[
|R_n(f; x)| \leq \frac{M_{n+1}}{(n+1)!} |w_n(x)|
\]
\[
\Rightarrow \text{/Огрубление оценки/} ~ |R_n(f; x)| \leq \frac{M_{n+1}}{(n+1)!} (b-a)^{n+1}
\]

\subsubsection*{Но:}
\begin{itemize}
    \item работает только для малых функций;
    \item сомнительная практическая ценность, если вспомнить ф-ую Рунге.
\end{itemize}

% Еще на листочке есть это, но я не помню, чтобы это рассказывалось и в конспектах ни у кого этого не видел
% \begin{enumerate}
%     \item При наличии одного узла кратности $m$, интерполяционный полином Эрмита становится многочленом Тейлора.
%     \textbf{проверить самим}
%     \item ---$\cdot \cdot$--- формула остатка совпадает с остатком в форме Лагранжа для ф-лы Тейлора.
%     \item Если все узлы простые, то формула остаточного члена совпадает с формулой погрешности простой интерполяции.
% \end{enumerate}

\subsection*{Интерполяционный процесс}

Имеет ли место сходимость интерполяционных полиномов к интерполируемой функции при неограниченном росте количества узлов?

\subsubsection*{Опр. (Интерполяционный процесс)}
Бесконечная треугольная матрица узлов:
\[
\begin{pmatrix}
x_0^{(0)} & 0 & \ldots \\
x_0^{(1)} & x_1^{(1)} & 0 & \ldots \\
x_0^{(2)} & x_1^{(2)} & x_2^{(2)} & 0 & \ldots \\
\vdots & \vdots & \vdots & \ddots
\end{pmatrix}, \quad x_i^{(n)} \in [a, b], ~ x_i^{(n)} \neq x_j^{(n)} \; \forall i \neq j
\]

На $[a, b]$ задан интерполяционный процесс, если элементы n-ой строки матрицы берутся в качестве узлов интерполяции, по которым строится последовательность интерполянтов $g_n(x), ~ n = 0,1,\ldots$

Если $g_n(x)$ — алгебраические полиномы, то алгебраический интерполяционный процесс.

\subsection*{Опр. (Сходимость интерполяционного процесса)}
Интерполяционный процесс для $f$ сходится в $y \in [a, b]$, если порождаемая им последовательность интерполянтов $g_n(x)$ сходится к $f(y)$ при $n \to \infty$. \\
---$\cdot \cdot$--- сходится равномерно, если $\max_{x \in [a, b]} |f(x) - P_n(x)| \xrightarrow[n \to \infty]{} 0$

\subsection*{Примеры расходимости}

\begin{enumerate}
    \item $f(x) = |x|$, $x \in [-1;1]$ --- равностоящие узлы \\
    Бернштейн: нет сходимости
    \item $f(x) = \frac{1}{1 + x^2}$, $x \in [-5, 5]$ --- равностоящие узлы \\
    Рунге: нет сходимости, $\lim_{n \to \infty} \max_{x \in [-1, 1]} |f(x) - P_n(x)| = \infty$
\end{enumerate}

\textbf{Но есть два известных результата: оптимистический и пессимистический:}

\subsection*{Теорема Марцинкевича}
Если функция непрерывна на интервале, то существует такая бесконечная треугольная матрица узлов из интервала, что соответствующий ей АИП для рассматриваемой функции сходится равномерно.

\subsection*{Теорема Фебера}
Не существует бесконечной треугольной матрицы узлов из заданного интервала, такой что соответствующий ей АИП сходился бы равномерно для любой непрерывной функции на интервале.

\subsection*{Неформальный вывод}
Алгебраические полиномы --- удобный, но достаточно непредсказуемый инструмент интерполирования, в том числе для довольно обычных, гладких функций.

\section*{Лекция 3.5}

\subsection*{Опр. (Полиномиальный сплайн)}
Пусть $a = x_0 < x_1 < \ldots < x_n = b$ — сетка на $[a, b]$. Полиномиальный сплайн на $[a, b]$ --- функция, которая на каждом отрезке $[x_{i-1}, x_i]$ является алгебраическим многочленом и на всём отрезке $[a, b]$ непрерывна вместе со своими производными до некоторого порядка.

\subsubsection*{Макс. степени полиномов --- степень сплайна}

\subsubsection*{Макс. номер разрыва --- гладкость сплайна}

\subsubsection*{Степень --- гладкость --- дефект}

\subsubsection*{Q: Что такое степень дефекта ноль?}
Один полином на всём отрезке области. \\
Дефект: "насколько сильно отличается от полинома"

\subsubsection*{Q: Как устроен простейший сплайн?}
Кусочно-линейная функция (Ст. 1, Деф. 1) \\
Ст. 2 --- параболические сплайны

\subsubsection*{Q: Откуда взялись кусочно полиномиальные функции?}
\begin{itemize}
    \item при описании реальности производные пятого порядка и выше --- редкость
    \item старшую производную: кусочно-линейная гиперполяция
    \item несколько интегрирований --- полиномы на отрезках
    \item получается кусочно-полиномиальная склейка
\end{itemize}

\subsubsection*{Q: Откуда такое странное название?}
Spline - гибкое лекало, использовали чертёжники для проведения гладких кривых через заданный набор точек. \\
Какая связь между сплайном и лекалом --- позже. \\
Первые работы --- М. Шёнберг 1946 г.; хотя ещё Лобачевский $\dots$

\subsubsection*{Q: Как задать сплайн?}
Пусть $d$ --- степень сплайна дефекта 1 --- дальше рассматриваем только такие. \\ $\{x_0, x_1, \ldots, x_n\}$ --- сетка.

\textbf{Требуется:} $n(d+1)$ значений коэффициентов, где n --- кол-во интервалов, d --- кол-во к-тов

\textbf{Имеем:}
\begin{itemize}
    \item $(n-1)d$ условий в узлах ($d-1$ производная), где n --- $x_1, \ldots, x_{n-1}$; d --- сплайн + d-1 производная
    \item $n+1$ усл. интерполяции (в $x_0, x_1, \ldots x_n$)
\end{itemize}

Всего --- $nd+n-d+1$

Ещё необходимо дополнительно задать $d-1$ условие, которое нужно  задавать дополнительно на концах, исходя из природы задачи.

Пусть $d \mathrel{\vdots} 2 \Rightarrow d-1 \not{\mathrel{\vdots}} ~~ 2 \Rightarrow$
\begin{enumerate}
    \item на разных концах интервала разное кол-во условий;
    \item интерполяционный сплайн чётной степени при некоторых естественных краевых условиях не существует.
\end{enumerate}

\textbf{Выходы:} \\
Выбор узлов сплайна между точками интерполяции:
\begin{itemize}
    \item подход Субботина, 1967 --- узлы сплайна посередине между заданными точками интерполяции;
    \item подход Марадена, 1974 --- заданная сетка узлов; точки интерполяции посередине.
\end{itemize}

"50 лет задаче Шёнберга о сходимости сплайн-интерполяции" (2014)

\subsubsection*{Кубический сплайн}

$d = 3$, дефект 1, (т. е. $C^2[a, b]$) --- узлы совпадают с точками интерполяции

\begin{itemize}
    \item нужно $4n$ к-тов
    \item имеем $3(n - 1) + (n + 1) = 4n - 2$ условий
\end{itemize}

\textbf{Ещё два:}

\begin{itemize}
    \item $S'(a) = \beta_0$, $S'(b) = \beta_n$ --- "наклон на концах" (I)
    \item $S''(a) = \gamma_0$, $S''(b) = \gamma_n$ --- "кривизна на концах" (II)
    \item $S^{(k)}(a) = S^{(k)}(b)$, $k = 0, 1, 2$ --- "гладкое продолжение за $[a;b]$" (III)
\end{itemize}

\textbf{Разобрать самим:} С.П.Шарый "Курс вычислительных методов"(2025 г.) глава 2.6б

\subsection*{Теорема (О погрешности интерполирования кубическими сплайнами)}
Пусть $p = 1, 2, 3, 4$; $f \in C^p[a, b]$ \\
$S(x)$ — интерполяционный кубический сплайн с краевыми условиями (I), (II), (III) на сетке $a = x_0 < x_1 < \ldots < x_n = b$ \\
Пусть $h = \max_i h_i = \max_i (x_i - x_{i-1})$ \quad // узлы интер. = узлы сплайна

Тогда для $k = 0, 1, 2 ~ (k \leq p )$ верно
\[
\max_{x \in [a, b]} |f^{(k)}(x) - S^{(k)}(x)| = O(h^{p-k})
\]

$\Rightarrow$ не будет проблем, как с фун-ей Рунге \quad // на любой равн. сетке интер. пр. сх. к непр. ф-ции

$\Rightarrow$ нужны оценки констант

Кривизна кривой определяется второй производной: \\
$S(x)$ --- естественный сплайн, если $S''(a) = S''(b) = 0$

\subsection*{Теорема (Холладея):}
Пусть $\mathcal{E}(f) = \int_a^b \left(f''(x)\right)^2 dx$; \\
Пусть $S(x)$ — естественный инт. кубический сплайн на $[a, b]$; \\
$\forall \varphi \in C^2[a, b]$, т. ч. $\varphi \neq S$, но $\varphi(x_i) = S(x_i)$, $i = 0, \ldots, n$ \\
\[
\Rightarrow \mathcal{E}(S) < \mathcal{E}(\varphi)
\]
\textbf{Смысл:} $\mathcal{E}(f) \approx$ энергия деформации упругой линейки, форма которой описывается $f(x)$. \\
\textbf{Краевые условия:} свободное закрепление на концах, без внешних сил. \\
$\Rightarrow$ Если упругая линейка закреплена в узлах интерполирования, то без воздействия внешних сил она примет форму, близкую к кубическому сплайну.

\begin{proof}
\[
(\varphi'')^2 - (s'')^2 = (\varphi'' - s'')^2 + 2s''(\varphi'' - s'')
\]
\[
\Rightarrow \mathcal{E}(\varphi) - \mathcal{E}(S) =\underbrace{\mathcal{E}(\varphi - s)}_{>0} + 2\int_a^b s''(\varphi'' - s'') dx
\]
\[
\int_a^b \underbrace{s''}_{u} \underbrace{(\psi'' - s'')}_{\partial v} dx = s''(\varphi' - s') \big|_a^b - \int_a^b s''' (\varphi' - s') dx
\]
\[
s''(\varphi' - s') \big|_a^b = \underbrace{s''(b)}_{=0} (\varphi' - s')(b) - \underbrace{s''(a)}_{=0} (\varphi' - s')(a) = 0
\]
\[
\int_a^b s''' (\varphi' - s') dx = C \int^b_a (\varphi' - s') dx = C \sum_{k=0}^{n-1} \int_{x_k}^{x_{k+1}} (\varphi - s)' dx = C \sum_{k=0}^{n-1} (\varphi - s) \big|_{x_k}^{x_{k+1}}=
\]
\[
= C \sum_{k=0}^{n-1} \big[(\varphi - s)(x_{k+1}) - (\psi - s)(x_k) \big] = 0
\]
\[
\Rightarrow \mathcal{E}(\varphi) - \mathcal{E}(S) = \mathcal{E}(\varphi - S) > 0
\]
\[
\Rightarrow \mathcal{E}(\varphi) > \mathcal{E}(S)
\]
\end{proof}

\subsubsection*{Трудности:}
\begin{itemize}
    \item необходимо решать СЛАУ;
    \item как добавлять/удалять узлы в реальном времени без перестроения.
\end{itemize}
